% Modul beinhaltet Befehl fuer Aufgabennummerierung,
% sowie die Header Informationen.

% Überschreibt enumerate Befehl, sodass 1. Ebene Items mit
\renewcommand{\theenumi}{(\alph{enumi})}
% (a), (b), etc. nummeriert werden.
\renewcommand{\labelenumi}{\text{\theenumi}}

% Counter für das Blatt und die Aufgabennummer.
% Ersetze die Nummer des Übungsblattes und die Nummer der Aufgabe
% den Anforderungen entsprechend.
% Beachte:
% \setcounter{countername}{number}: Legt den Wert des Counters fest
% \stepcounter{countername}: Erhöht den Wert des Counters um 1.
\newcounter{sheetnr}
\newcounter{exnum}
\newcounter{exexnum}
\setcounter{exnum}{0}

\newcommand{\newsheet}[1]{%
    \setcounter{exnum}{0}
    \setcounter{sheetnr}{#1}
}

% Nutze den \exercise{Aufgabenname} Befehl, um eine neue Aufgabe zu beginnen.
\newcommand{\exercise}[2]{%
    \stepcounter{exnum}% erhöhe exnum um 1 
    \section*{{\theexnum} {#1} \printpoints{#2}}%
    \setcounter{exexnum}{0}% Unteraufgabe zurücksetzen
}

% Nutze den \setexercise{Unteraufgabenname} Befehl, um eine neue Unteraufgabe zu beginnen.
\newcommand{\subexercise}[2]{% Befehl für Unteraufgabentitel
    \stepcounter{exexnum}% erhöhe exexnum um 1 
    \subsection*{{\theexnum.\theexexnum} {#1} \printpoints{#2}}
}

\newcommand{\printpoints}[1]{%
  \ifnum#1=0
    % nichts
  \else
    \ifnum#1=1
      \hfill 1 point
    \else
      \hfill #1 points
    \fi
  \fi
}

\lstset{
  basicstyle=\ttfamily\small,
  keywordstyle=\color{blue},
  stringstyle=\color{red},
  commentstyle=\color{gray},
  frame=single,
  breaklines=true,
  showstringspaces=false,
  captionpos=b
}

% Formatierung der Kopfzeile
% \ohead: Setzt rechten Teil der Kopfzeile mit
% Namen und Matrikelnummern aller Bearbeiter
\ohead{Ammar Elbashir (3721991)\\
Ethan Banovic (3724710)\\
Arda Sarier (3721030)}

% \chead{} kann mittleren Kopfzeilen Teil sezten
% \ihead: Setzt linken Teil der Kopfzeile mit
% Modulnamen, Semester und Übungsblattnummer
\ihead{Foundations of Artificial Intelligence \\
Winter Term 2025/26\\
Homework \thesheetnr}